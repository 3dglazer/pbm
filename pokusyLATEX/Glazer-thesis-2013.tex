%% History:
% Pavel Tvrdik (26.12.2004)
%  + initial version for PhD Report
%
% Daniel Sykora (27.01.2005)
%
% Michal Valenta (3.12.2008)
% rada zmen ve formatovani (diky M. Duškovi, J. Holubovi a J. Žďárkovi)
% sjednoceni zdrojoveho kodu pro anglickou, ceskou, bakalarskou a diplomovou praci

% One-page layout: (proof-)reading on display
%%%% \documentclass[11pt,oneside,a4paper]{book}
% Two-page layout: final printing
\documentclass[11pt,twoside,a4paper]{book}   
%\documentclass [a4paper,11pt,twoside,notitlepage,openright]{report} %print on both side
%\usepackage[bookmarks=false, colorlinks=false,unicode]{hyperref}
\usepackage{fancyhdr}
\usepackage{amsmath}
\usepackage{amssymb}
\usepackage[amsmath,thmmarks]{ntheorem}
\usepackage{latexsym}
\usepackage{float}
\usepackage[width=.95\textwidth,small]{caption}
\usepackage{subfigure}                  %text beside figure
\usepackage{graphicx}
\usepackage{epstopdf}
%=-=-=-=-=-=-=-=-=-=-=-=--=%
% The user of this template may find useful to have an alternative to these 
% officially suggested packages:
\usepackage[czech, english]{babel}
\usepackage[T1]{fontenc} % pouzije EC fonty 
% pripadne pisete-li cesky, pak lze zkusit take:
% \usepackage[OT1]{fontenc} 
\usepackage[utf8]{inputenc}
%=-=-=-=-=-=-=-=-=-=-=-=--=%
% In case of problems with PDF fonts, one may try to uncomment this line:
%\usepackage{lmodern}
%=-=-=-=-=-=-=-=-=-=-=-=--=%
%=-=-=-=-=-=-=-=-=-=-=-=--=%
% Depending on your particular TeX distribution and version of conversion tools 
% (dvips/dvipdf/ps2pdf), some (advanced | desperate) users may prefer to use 
% different settings.
% Please uncomment the following style and use your CSLaTeX (cslatex/pdfcslatex) 
% to process your work. Note however, this file is in UTF-8 and a conversion to 
% your native encoding may be required. Some settings below depend on babel 
% macros and should also be modified. See \selectlanguage \iflanguage.
%\usepackage{czech}  %%%%%\usepackage[T1]{czech} %%%%[IL2] [T1] [OT1]
%=-=-=-=-=-=-=-=-=-=-=-=--=%

%%%%%%%%%%%%%%%%%%%%%%%%%%%%%%%%%%%%%%%
% Styles required in your work follow %
%%%%%%%%%%%%%%%%%%%%%%%%%%%%%%%%%%%%%%%
%\usepackage{indentfirst} %1. odstavec jako v cestine.

\usepackage{k336_thesis_macros} % specialni makra pro formatovani DP a BP
 % muzete si vytvorit i sva vlastni v souboru k336_thesis_macros.sty
 % najdete  radu jednoduchych definic, ktere zde ani nejsou pouzity
 % napriklad: 
 % \newcommand{\bfig}{\begin{figure}\begin{center}}
 % \newcommand{\efig}{\end{center}\end{figure}}
 % umoznuje pouzit prikaz \bfig namisto \begin{figure}\begin{center} atd.


%%%%%%%%%%%%%%%%%%%%%%%%%%%%%%%%%%%%%
% Zvolte jednu z moznosti 
% Choose one of the following options
%%%%%%%%%%%%%%%%%%%%%%%%%%%%%%%%%%%%%
%\newcommand\TypeOfWork{Diplomová práce} \typeout{Diplomova prace}
\newcommand\TypeOfWork{Master's Thesis}   \typeout{Master's Thesis} 
% \newcommand\TypeOfWork{Bakalářská práce}  \typeout{Bakalarska prace}
% \newcommand\TypeOfWork{Bachelor's Project}  \typeout{Bachelor's Project}


 \newcommand\StudProgram{Open Informatics}  %master program
\newcommand\StudBranch{Computer Graphics}
%%%%%%%%%%%%%%%%%%%%%%%%%%%%%%%%%%%%%%%%%%%%
% Vyplnte nazev prace, autora a vedouciho
% Set up Work Title, Author and Supervisor
%%%%%%%%%%%%%%%%%%%%%%%%%%%%%%%%%%%%%%%%%%%%

\newcommand\WorkTitle{Progressive Computation of Global Illumination}
\newcommand\FirstandFamilyName{Bc. Zdeněk Glazer}
\newcommand\Supervisor{Ing. Jaroslav Sloup}


% Pouzijete-li pdflatex, tak je prijemne, kdyz bude mit vase prace
% funkcni odkazy i v pdf formatu
\usepackage[
pdftitle={\WorkTitle},
pdfauthor={\FirstandFamilyName},
bookmarks=true,
colorlinks=true,
breaklinks=true,
urlcolor=red,
citecolor=blue,
linkcolor=blue,
unicode=true,
]
{hyperref}

%------<<<<<<<<<< CUSTOM COMMANDS DEFINITIONS >>>>>>>>>>------------
\newcommand{\myFigure}[5]{
  \begin{figure}
    \centering
    \includegraphics[width=#1\linewidth]{#2}
    \caption[#3]{#4}\label{#5}
  \end{figure}
}


\newcommand{\ita}[1]{
\textit{#1}
}

\newcommand{\bd}[1]{
\textbf{#1}
}


%-----<<<<<<<<<<<< NAMES NAD PATHS >>>>>>>>>>>>-----
\def \BookName {Masterer Thesis}
\def \Bookname {Progressive Computation of Global Illumination}
\def \Authors {Author\?: Bc. Zdeněk Glazer}
\def \DatumDP {Prague, 2012}

\def \CVUT {Czech Technical University in Prague}
\def \FEL {Faculty of Electrical Engineering}
\def \DCE {Department of Computer Graphics and Interaction}

\def \cestaFiles {00_Chapters/}
\def \cestaDva {02_Chap2/}
\def \cestaTri {03_Chap3/}
\def \cestaFour{04_SelectedAlgos/}
\def \cestaFive{05_Proposedsolution/}
\def \cestaSix{06_Implementation/}
\def \cestaSeven{07_Results/}
\def \cestaCtyri {04_Concl/}
\def \cestaAP {06_Appendix/}
\def \cestaImg{images/}
%-----<<< --------------------------------- >>>-----

\begin{document}

%%-----<<< HEAD >>>-----
%\pagestyle{empty}                       %no pagination
%\BookHeadDP
%\cleardoublepage

%%-----<<< ---- >>>-----
%%%%%%%%%%%%%%%%%%%%%%%%%%%%%%%%%%%%%
% Zvolte jednu z moznosti 
% Choose one of the following options
%%%%%%%%%%%%%%%%%%%%%%%%%%%%%%%%%%%%%
%\selectlanguage{czech}
\selectlanguage{english} 

\iflanguage{czech}{
	 \typeout{************************************************}
	 \typeout{Zvoleny jazyk: cestina}
	 \typeout{Typ prace: \TypeOfWork}
	 \typeout{Studijni program: \StudProgram}
	 \typeout{Obor: \StudBranch}
	 \typeout{Jmeno: \FirstandFamilyName}
	 \typeout{Nazev prace: \WorkTitle}
	 \typeout{Vedouci prace: \Supervisor}
	 \typeout{***************************************************}
	 \newcommand\Department{Katedra počítačů}
	 \newcommand\Faculty{Fakulta elektrotechnická}
	 \newcommand\University{České vysoké učení technické v Praze}
	 \newcommand\labelSupervisor{Vedoucí práce}
	 \newcommand\labelStudProgram{Studijní program}
	 \newcommand\labelStudBranch{Obor}
}{
	 \typeout{************************************************}
	 \typeout{Language: english}
	 \typeout{Type of Work: \TypeOfWork}
	 \typeout{Study Program: \StudProgram}
	 \typeout{Study Branch: \StudBranch}
	 \typeout{Author: \FirstandFamilyName}
	 \typeout{Title: \WorkTitle}
	 \typeout{Supervisor: \Supervisor}
	 \typeout{***************************************************}
	 \newcommand\Department{Department of Computer Science and Engineering}
	 \newcommand\Faculty{Faculty of Electrical Engineering}
	 \newcommand\University{Czech Technical University in Prague}
	 \newcommand\labelSupervisor{Supervisor}
	 \newcommand\labelStudProgram{Study Programme} 
	 \newcommand\labelStudBranch{Field of Study}
}

%%%%%%%%%%%%%%%%%%%%%%%%%%    Poznamky ke kompletaci prace
% Nasledujici pasaz uzavrenou v {} ve sve praci samozrejme 
% zakomentujte nebo odstrante. 
% Ve vysledne svazane praci bude nahrazena skutecnym 
% oficialnim zadanim vasi prace.
%-----<<< ZAD¡NÕ DIPLOMOV… PR¡CE >>>-----
\pagenumbering{roman} \cleardoublepage \thispagestyle{empty}
\chapter*{Na tomto místě bude oficiální zadání vaší práce}
\begin{itemize}
\item Toto zadání je podepsané děkanem a vedoucím katedry,
\item musíte si ho vyzvednout na studiijním oddělení Katedry počítačů na Karlově náměstí,
\item v jedné odevzdané práci bude originál tohoto zadání (originál zůstává po obhajobě na katedře),
\item ve druhé bude na stejném místě neověřená kopie tohoto dokumentu (tato se vám vrátí po obhajobě).
\end{itemize}
\cleardoublepage
%-----<<< ---------------------- >>>-----

%%%%%%%%%%%%%%%%%%%%%%%%%%    Titulni stranka / Title page 

\coverpagestarts
\cleardoublepage
%%%%%%%%%%%%%%%%%%%%%%%%%%%    Podekovani / Acknowledgements 


%-----<<< ACKNOWLEDGEMENT >>>-----
\input{\cestaFiles Acknow.tex}          %input file
\cleardoublepage
%-----<<< --------------- >>>-----

   %-----<<< DECLARATION >>>-----
\pagestyle{plain}                       %no pagination
\pagenumbering{roman}                   %start roman pagination from 1
\input{\cestaFiles Declar.tex}          %input file
\newpage
\cleardoublepage
%-----<<< ----------- >>>-----

%-----<<< ABSTRACT >>>-----
\input{\cestaFiles Abstract.tex}        %input file
\cleardoublepage
%-----<<< -------- >>>-----


   
%-----<<< TABLE OF CONTENTS >>>-----
\setcounter{secnumdepth}{4}             %number of section to 4
\setcounter{tocdepth}{4}                %number of section in table of contents greater then 3
\tableofcontents
\cleardoublepage
%-----<<< ----------------- >>>-----


%-----<<< TABLE OF FIGURES >>>-----
\addcontentsline{toc}{chapter}{List of figures}
\listoffigures
\cleardoublepage
%-----<<< ---------------- >>>-----


%-----<<< TABLE OF TABLES >>>-----
\addcontentsline{toc}{chapter}{List of tables}
\listoftables
\cleardoublepage
%-----<<< --------------- >>>-----


%-----<<< TABLE OF CONTENTS PAGINATION >>>-----
\pagenumbering{arabic}                  %start arabic pagination from 1
%-----<<< ---------------------------- >>>-----


%-----<<< CHAPTERS >>>-----
\hyphenation{Automatica}                %no divide words
%zakomentoval Jsem oproti puvodnimu
%\pagestyle{headings}

\mainbodystarts
% horizontalní mezera mezi dvema odstavci
%\parskip=5pt
%11.12.2008 parskip + tolerance
\normalfont
\parskip=0.2\baselineskip plus 0.2\baselineskip minus 0.1\baselineskip

% Odsazeni prvniho radku odstavce resi class book (neaplikuje se na prvni 
% odstavce kapitol, sekci, podsekci atd.) Viz usepackage{indentfirst}.
% Chcete-li selektivne zamezit odsazeni 1. radku nektereho odstavce,
% pouzijte prikaz \noindent.

%**************************************************************

% Pro snadnejsi praci s vetsimi texty je rozumne tyto rozdelit
% do samostatnych souboru nejlepe dle kapitol a tyto potom vkladat
% pomoci prikazu \include{jmeno_souboru.tex} nebo \include{jmeno_souboru}.
% Napr.:
% \include{1_uvod}
% \include{2_teorie}
% atd...


\chapter{Introduction}
Problem definition. What is optically active environment and example images. What is challenging. 

Why global illumination and why progressive methods

%\begin{figure}
%   \centering
%   \includegraphics{example.jpg} % requires the graphicx package
%   \caption{example caption}
%   \label{fig:example}
%\end{figure}
%
%   
%Co to je rendering
%Zobrazovaci rovnice distribuce energie v prostoru. 
%
%Co to je raytracing
%paprsek protina geometrii v zakladu 
%
%Jsou stale poupularnejsi progresivni metody renderingu, kdy se obraz pred stale vylepsuje, takze muzem videt nahled velice rychle.
%
%ray tracing se stava poupularnejsi a popularnejsi, mene narocny na nastavovani a fizikalne korektni. Navic sceny jsou slozitejsi a slozitejsi divaci narocnejsi. 
%
%Co to jsou volumetricke efekty a opticky aktivni prostredi
%
%Avsak volumetricke efekty se stale dost casto rasterizujou.
%
%Jak je to náročný a že se stále dost používají aproximační metody (rasterizace, deep shadow maps ) a nemodelují se náročné věci jako radiative transport surface to media volumetric caustics atd. Ani nástroje používané v produkci přímo nepodporuje a nebo jen v podobě náročných fotonových map, je to pomalé , takže artisti nemůžou odladit vse co chteji malo iteraci.



%\section{Podkapitola}
              % input fiel
\input{\cestaDva Chap2.tex} 		% Fundaments of realistic image synthesis
\input{\cestaTri Chap3.tex}		% Common solutions
\input{\cestaFour Chap4.tex}		% Consistant progressive methods
\input{\cestaFive Chap5.tex}		% Proposed solutions
\input{\cestaSix Chap6.tex}		% Implementation
\input{\cestaSeven Chap7.tex}		% Results

\input{\cestaCtyri Concl.tex}		% Conclusion
%-----<<< -------- >>>-----


%-----<<< REFERENCES >>>-----
%\begin{thebibliography}{xx}

% ======== reference k progressive beam mappingu atd =====
@article{novak12vbls,
    author = {Jan Nov{\'a}k and Derek Nowrouzezahrai and Carsten Dachsbacher and Wojciech Jarosz},
    title = {Progressive Virtual Beam Lights},
    journal = {Computer Graphics Forum (Proceedings of EGSR 2012)},
    volume = {31},
    number = {4},
    year = {2012},
    month = jun,
    keywords = {photon beams, photon mapping, final gather, participating media, VPL, virtual point lights, VSL, virtual spherical lights, indirect illumination, weak singularity}
}

@article{jarosz11comprehensive,
    author = {Wojciech Jarosz and Derek Nowrouzezahrai and Iman Sadeghi and Henrik Wann Jensen},
    title = {A Comprehensive Theory of Volumetric Radiance Estimation Using Photon Points and Beams},
    journal = {ACM Transactions on Graphics (Presented at ACM SIGGRAPH 2011)},
    volume = {30},
    number = {1},
    year = {2011},
    month = jan,
    pages = {5:1--5:19},
    keywords = {photon beams, photon mapping, beam radiance estimate, density estimation, participating media}
}

@article{novak12vrls,
    author = {Jan Nov{\'a}k and Derek Nowrouzezahrai and Carsten Dachsbacher and Wojciech Jarosz},
    title = {Virtual Ray Lights for Rendering Scenes with Participating Media},
    journal = {ACM Transactions on Graphics (Proceedings of ACM SIGGRAPH 2012)},
    volume = {31},
    number = {4},
    year = {2012},
    month = jul,
    keywords = {photon beams, photon mapping, final gather, participating media, VPL, virtual point lights, indirect illumination, weak singularity, unbiased}
}

@article{jarosz11progressive,
    author = {Wojciech Jarosz and Derek Nowrouzezahrai and Robert Thomas and Peter-Pike Sloan and Matthias Zwicker},
    title = {Progressive Photon Beams},
    journal = {ACM Transactions on Graphics (Proceedings of ACM SIGGRAPH Asia 2011)},
    volume = {30},
    number = {6},
    year = {2011},
    month = dec
}

@article{jarosz08beam,
    author = {Wojciech Jarosz and Matthias Zwicker and Henrik Wann Jensen},
    title = {The Beam Radiance Estimate for Volumetric Photon Mapping},
    journal = {Computer Graphics Forum (Proceedings of Eurographics 2008)},
    volume = {27},
    number = {2},
    year = {2008},
    month = apr,
    pages = {557--566}
}

% ========End of: reference k progressive beam mappingu atd =====



% ======== reference k progressive photon mappingu

%========End of:  reference k progressive photon mappingu

%======== neco k volume renderingu sampling + deep shadow mapping, furier maps atd.

%======== End of:  neco k volume renderingu sampling + deep shadow mapping, furier maps atd.



\end{thebibliography}



\bibliographystyle{alphaurl}
\bibliography{mybibliography}

%\bibliographystyle{Styles/Skripta}
%\bibliography{Styles/Refer}                   %references from BIBTEX
\addcontentsline{toc}{chapter}{Literatura}
%-----<<< ---------- >>>-----

%-----<<< APPENDIXS >>>-----
\cleardoublepage
\def\appendixname{Appendinx}
\pagenumbering{Roman}                   %start arabic pagination from 1
\begin{appendix}
\input{\cestaAP Appen1.tex}             %input file
\input{\cestaAP Appen2.tex}             %input file
%\input{\cestaAP Appen3.tex}             %input file
\end{appendix}
%-----<<< --------- >>>-----

   
   
  \end{document}