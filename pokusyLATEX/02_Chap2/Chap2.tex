\chapter{Fundamentals of realistic image synthesis}

In this chapter we briefly revise fundamental theory behind realistic computer generated imagery. Our aim is to synthesize (render) image of a virtual scene. To do this properly, with physical accuracy in mind, we have to find a suitable physical model of light.
\\
\\
If we would like to be absolutely correct, we would have to use quantum optics model. This would lead us to tracing individual photons and interaction with matter on an atomic level, which is completely computationally unfeasible. Nevertheless when it comes to computer graphics even a simplified ray optics model can handle majority of the phenomena we are interested in. This model assumes that light travels in straight lines, at infinite speed and the only things that might happen to it are emission, absorption, reflection, and transmission. Using this model we will be unable to simulate phenomena such as diffraction (for example chromatic aberration), polarization and other similar effects depending on light wavelengths and polarity.
\\
\\
When the suitable light model is chosen, we should define some numerical quantities we can compute with. In the next section we will start with radiometry quantities and inspect problem of light interaction with surfaces. This gives us enough fundaments to formulate rendering equation. We continue with light interaction with participating media, which leads us to extended rendering equation. Algorithms mentioned in the next chapters are based on this theory and they present various methods solving this complex equation.
\\
\\


\section{Radiometry}

Radiometry is set of techniques for measuring different kinds of electromagnetic radiation such as light. We could have used a photometry, which on the other hand measures the response of a human eye to visible light. This basically corresponds to measurements retrieved using radiometry and weighted with an eye response function. For the purposes of this thesis we will use radiometric quantities, because corresponding photometric quantities can be easily derived.

\subsection{Radiometric quantities}
\bd{Radiant flux}, also known as radiant power, denoted as $\phi$.
\begin{equation}
\phi=\frac{dQ}{dt}\;\;[W=J.s^{-1}]
\end{equation}
%prehodit vety a pridat 
Where Q [J] denotes radiant energy, which describes how much energy (amount of photons times their energy) is presented in a given area at a point in time. This leads us to flux definition as  amount of radiant energy going through a location over time , alternatively speaking how fast is the change of the amount of the photons in a specified location. 
\\
\\
\bd{Irradiance}, also known as flux density, denoted as \ita{E(x).} 
\begin{equation}
E(x)=\frac{d\phi(x)}{dA}\;\;[W.m^{-2}]
\end{equation}
The irradiance at a point \ita{x} on the surface \ita{S,} can be expressed as an amount of incident flux to differential area at a given time.
\\
\\
\bd{Radiance} is probably the most important quantity. Denoted as $L(x,\omega)$, where x is a point of interest and $\omega$ is a differential angle in a given direction, also known as solid angle.
\begin{equation}
L(x,\omega)=\frac{d\phi(x)}{cos\theta dAd \omega}\;\;[W.m^{-2}.sr^{-1}]
\end{equation}
Radiance expresses how much flux is coming from a differential direction $d\omega$ onto a differential surface $dA$. The $cos\theta$ factor compensates the shortening amount of irradiance on the surface with growing $\theta$\footnote{The meaning of $\theta$ is $cos(\theta)=(\vec{n}.\vec{\omega})$, where $\vec{n}$ is surface normal at a point of interest (x) and $\vec{\omega}$ is a direction vector we want to compute radiance for.}, while the level of illumination is maintained. This is the quantity, which has to be gathered for every pixel in a rendered image. Furthermore other quantities can be computed from radiance using integration. 

\subsection{Relationships between quantities}
Tady tabulka nejzasadnejsich quantities a jejich jednotek 

%\begin{table}
    \begin{tabular}
        Radiant energy & $Q$    & [J]                \\ 
        Radiant flux   & $\phi$ & [W]                \\ 
        %Irradiance     & $E$    & [W.m^{-2}]         \\ 
       % Radiance       & $L$    & [W.m^{-2}.sr^{-1}] \\
    \end{tabular}
%\end{table}


% great page for latex equation typing http://www.codecogs.com/latex/eqneditor.php
Quasi Monte Carlo equation
\begin{equation*}
\int _{I^{S}}f(x)dx\approx\frac{1}{N}\sum_{i=1}^{N}f(x_{i})),
\end{equation*}
%\int _{I^{S}}f(x)dx\approx\frac{1}{N}\sum_{i=1}^{N}f(x_{i})),

\cite{Dutre01globalillumination} %global illuminaiton compendium.
1) what is radiometry
- asi z Jarosze + wikipedia

2) quantities:
- z meho prehledu z rso
flux: 
irradiance:
radiance:

Relationships between named quantities.
-prehled rso

\section{Surface interaction}
-advanced illum techniques

assumptions and symplificaitions 
\subsection{BRDF}
-pbrt
\subsection{Rendering equation}
-advanced illum techniques
Theory behind rendering equation, integration of radiometry quantities. Both surface and solid angle deffinisions

\section{Volume interaction}
assumptions made of particles simplification using probability
-advanced illum techniques
-popsat absorbci a dalsi veci
\subsection{Phase functions}
\myFigure{12}{02_Chap2/Figures/heneygreenstein2dgraph}{2D plot of Plot of Heney-Greenstein phase function.}{Plot of Heney-Greenstein phase function with different g \\ coefficients. For g\textless0 the function represents backscattering media, for g=0 isometric scattering and for g\textgreater0 growing \\forward scattering tendencies.}



-isotropic
-anisotropic name hg,shlick aprox, reylight (atmosphere)
-jak dochazi k rozptylu svetla
-pouzit matlabacky obrazky heney-g funkce ruzne koeficienty

\subsection{Rendering equation including participating media}
hezke odvozeni je v -advanced illum techniques

What steps have to be taken to get rendering equation to the next level.

\section{Volume and surface interaction}
In order to properly simulate light transport in the scenes containing both surfaces and volumetrically active media.

Different light transport paths surface surface, surface media, media surface, media media.

\subsection{Extended rendering equation}
Put together all the integrals
pokracovat v odvozeni z -advanced illum techniques


%\section{Images and tables}

%\subsection{Images}


