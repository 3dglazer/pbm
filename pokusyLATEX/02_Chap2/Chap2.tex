\chapter{Fundamentals of realistic image synthesis}

In this chapter we briefly revise fundamental theory behind realistic computer generated imagery. We first describe basic radiometry quantities and inspect problem of light interaction with surfaces. This gives us enough fundaments to formulate rendering equation. We continue with light interaction with participating media, which leads us to extended rendering equation. Algorithms mentioned in the next chapters are based on this theory and they present various methods solving this complex equation.

\section{Radiometry}

% great page for latex equation typing http://www.codecogs.com/latex/eqneditor.php
Quasi Monte Carlo equation
%\int _{I^{S}}f(x)dx\approx\frac{1}{N}\sum_{i=1}^{N}f(x_{i})),

\cite{Dutre01globalillumination} %global illuminaiton compendium.
1) what is radiometry
- asi z Jarosze + wikipedia

2) quantities:
In this chapter I will define basic radiometry quantities to build enough support knowledge for the next chapter. 
- z meho prehledu z rso
flux: 
irradiance:
radiance:

Relationships between named quantities.
-prehled rso

\section{Surface interaction}
-advanced illum techniques

assumptions and symplificaitions 
\subsection{BRDF}
-pbrt
\subsection{Rendering equation}
-advanced illum techniques
Theory behind rendering equation, integration of radiometry quantities. Both surface and solid angle deffinisions

\section{Volume interaction}
assumptions made of particles simplification using probability
-advanced illum techniques
-popsat absorbci a dalsi veci
\subsection{Phase functions}
\myFigure{12}{02_Chap2/Figures/heneygreenstein2dgraph}{2D plot of Plot of Heney-Greenstein phase function.}{Plot of Heney-Greenstein phase function with different g \\ coefficients. For g\textless0 the function represents backscattering media, for g=0 isometric scattering and for g\textgreater0 growing \\forward scattering tendencies.}



-isotropic
-anisotropic name hg,shlick aprox, reylight (atmosphere)
-jak dochazi k rozptylu svetla
-pouzit matlabacky obrazky heney-g funkce ruzne koeficienty

\subsection{Rendering equation including participating media}
hezke odvozeni je v -advanced illum techniques

What steps have to be taken to get rendering equation to the next level.

\section{Volume and surface interaction}
In order to properly simulate light transport in the scenes containing both surfaces and volumetrically active media.

Different light transport paths surface surface, surface media, media surface, media media.

\subsection{Extended rendering equation}
Put together all the integrals
pokracovat v odvozeni z -advanced illum techniques


%\section{Images and tables}

%\subsection{Images}


