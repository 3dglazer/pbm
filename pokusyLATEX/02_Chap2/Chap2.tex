\chapter{Fundamentals of realistic image synthesis}

\section{Radiometry}
1) what is radiometry
- asi z Jarosze + wikipedia

2) quantities:
In this chapter I will define basic radiometry quantities to build enough support knowledge for the next chapter. 
- z meho prehledu z rso
flux: 
irradiance:
radiance:

Relationships between named quantities.
-prehled rso

\section{Surface interaction}
-advanced illum techniques

assumptions and symplificaitions 
\subsection{BRDF}
-pbrt
\subsection{Rendering equation}
-advanced illum techniques
Theory behind rendering equation, integration of radiometry quantities. Both surface and solid angle deffinisions

\section{Volume interaction}
assumptions made of particles simplification using probability
-advanced illum techniques
-popsat absorbci a dalsi veci
\subsection{Phase functions}
-jak dochazi k rozptylu svetla
-pouzit matlabacky obrazky heney-g funkce ruzne koeficienty

\subsection{Rendering equation including participating media}
hezke odvozeni je v -advanced illum techniques

What steps have to be taken to get rendering equation to the next level.

\section{Volume and surface interaction}
In order to properly simulate light transport in the scenes containing both surfaces and volumetrically active media.

Different light transport paths surface surface, surface media, media surface, media media.

\subsection{Extended rendering equation}
Put together all the integrals
pokracovat v odvozeni z -advanced illum techniques


%\section{Images and tables}

%\subsection{Images}


