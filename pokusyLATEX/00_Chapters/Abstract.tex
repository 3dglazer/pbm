%%%%%%%%%%%%%%%%%%%%%%%%%%%%    Abstract 
 
\abstractpage
\label{abstract}
Global illumination always played a key role in realistic computer generated image synthesis. Many algorithms have been developed to compute realistic images in reasonable time, though rendering volumetric phenomenas such as smoke including volumetric caustics and complex glossy reflection and refraction paths is still consuming task. Recently some new light caching approaches and progressive rendering techniques were discovered. This thesis aims to test their accuracy and feasibility on various scenes containing heterogenous participating media. 

% Prace v cestine musi krome abstraktu v anglictine obsahovat i
% abstrakt v cestine.
\vglue60mm

\noindent{\Huge \textbf{Abstrakt}}
\vskip 2.75\baselineskip

\noindent
Globální osvětlení vždy hrálo klíčovou roli při realistické syntéze obrazu. Existuje mnoho algoritmů které dokáží realistický obraz vypočítat v rozumném čase, ale výpočet globálního osvětlení u scén obsahujících opticky aktivní prostředí, jako je například kouř je stále časově velice náročný. V nedávné době se objevily nové postupy předvýpočtu světelných cest ve scéně a nové progresivní metody výpočtu obrazu.V této diplomové práci chceme prověřit jejich přesnost a rychlost na různých scénách obsahujicích nehomogenní opticky aktivní prostřědí.


