%%%%%%%%%%%%%%%%%%%%%%%%%%%%    Abstract 
 
\abstractpage
\label{abstract}
Global illumination always played a key role in realistic computer generated image synthesis. Many algorithms have been developed to compute realistic images in reasonable time, though rendering volumetric phenomena such as smoke including volumetric caustics, complex glossy reflection and refraction paths is still time demanding task. Recently some new radiance caching approaches and progressive rendering techniques were discovered. This thesis aims to test accuracy and feasibility of selected progressive rendering techniiques on various scenes containing heterogenous participating media. 

% Prace v cestine musi krome abstraktu v anglictine obsahovat i
% abstrakt v cestine.
\vglue60mm

\noindent{\Huge \textbf{Abstrakt}}
\vskip 2.75\baselineskip

\noindent
Globální osvětlení vždy hrálo klíčovou roli při realistické syntéze obrazu. Existuje mnoho algoritmů které dokáží realistický obraz vypočítat v rozumném čase, ale výpočet globálního osvětlení u scén obsahujících opticky aktivní prostředí, jako je například kouř je stále časově velice náročný. V nedávné době byly publikovány nové postupy předvýpočtu záře ve scéně a nové progresivní metody výpočtu obrazu.V této diplomové práci porovnáme přesnost a rychlost vzbraných progresivních metod na různých scénách obsahujicích nehomogenní opticky aktivní prostřědí.


