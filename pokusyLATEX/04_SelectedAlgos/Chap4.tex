\chapter{Consistant progressive methods}
what does it mean to be progressive
\section{Stochastic progressive photon mapping}
Progressive photon mapping(PPM) was first introduced by Toshiya Hachisuka, Shinji Ogaki and Henrik Wann Jensen in  \cite{Hachisuka:2008:PPM} article and later on extended to Stochastic progressive photon mapping (SPPM) in \cite{Hachisuka:2009:SPP}. Both methods share the same photon shooting step with photon mapping. The difference is in the photon map radiance evaluation.
\\
\\
PPM samples the scene only once using distributed ray-tracing storing ray intersection
points in the acceleration structure in the first pass. In the next passes, when the photons
are shoot into the scene insted of saving them to the photon map, the intersection points
in a given radius are querried and radiance values in the corresponding image samples are
updated. Down side of this approach is that there are quite many intersections to be stored.
Extended algorithm with volumetric photon mapping capability would havve to sore all
transmittance evaluation points to, which would be very memory intensive.
\\
\\
The SPPM approach raytraces the scene again for every new progressive pass (see figure
\ref{fig::SPPMPPM}). This way it trades of some extra calculation for memory footprint. The extra ray
samples might be very usefull when it comes to stochastic effects, such ass depth of field
simulation or motion blur, because many rays with different spaciotemporal samplig have to
be generated anyway for noise free result.
\\
\\
In both cases multiple photon shooting passes are used with progressively reducing photon
radius. The radius reduction ensures that the bias goes to zero in the limit.
\\
\\
 =========== TODO REDUCTION RADIUS RATE EQUATION
\myFigure{1.}{images/ppmsppm}{Progressive photon mapping and stochastic progressive photon mapping comparison. Source \cite{Hachisuka:2009:SPP}.}{Progressive photon mapping and stochastic progressive photon mapping comparison. Source \cite{Hachisuka:2009:SPP}.}{fig::SPPMPPM}

\myFigure{0.5}{images/vpl2vbl}{Comparison vpl, vsl, vrl, vbl.}{During the preprocessing stage photon paths are shoot into the participating media. This image compares the methods we can store and evaluate cached radiance in the form of virtual lights. This figure demonstrates the analogy of virtual spherical lights to virtual point lights with virtual beam lights to virtual ray lights. Source \cite{novak12vbls}.}{fig:vpl2vbl}
\section{Virtual point lights}
\cite{Keller97instantradiosity} %original paper about vpls 

\section{Virtual spherical lights}
\cite{HasanVSL} %virtual spherical lights


\section{Beam mapping}
\cite{jarosz11progressive}
Hybrid solution cpu gpu rasterization..



\section{Virtual ray lights}
\cite{novak12vrls}

\section{Virtual beam lights}
As you can se on the image fig \ref{fig:vpl2vbl}
\cite{novak12vbls}

%\section{Images and tables}

%\subsection{Images}


