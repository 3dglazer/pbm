\chapter{Common solutions}
In this chapter we will try to summarize the most common technics for solving the rendering equation in the presence of participating media.

\section{ Rasterization}
Films - 

krakatoa rendering engine
\myFigure{8}{\cestaImg nvidiasmokeparticles.jpg}{Nvidia smoke particle demo.}{This image has been rendered using multiple shadow maps for direct illumination on gpu.}

nvidia smoke particles demo 
pixar deep shadow maps \cite{LokDSM}

Still most used in films and games - fast approximation.
Particle rasterization + rendering slices of 3D volumes can solve only direct illumination.
\section{Raytracing}
\subsection{Raymarching volumes}
More recently, another technique has become popular for sampling distances along a ray in an inhomogeneous medium. The idea comes from the neutron transport community in the 60s and has various names (delta tracking, pseudo-scattering); we’ll call it Woodcock tracking in effort to promote original paper.


pbrt single scattering
can support procedural volumes and any type of lighting
\subsection{Unbiased methods}
Plus and cons - very slow no caching recomputation of visibility factor
\subsection{Biased methods}
\cite{jarosz08thesis} %jarosz thesis irradiance caching
irradiance caching, Photon tracing, final gather ...
Plus and cons

