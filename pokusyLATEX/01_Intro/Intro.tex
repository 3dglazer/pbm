\chapter{Introduction}
Problem definition. What is optically active environment and example images. What is challenging. 

Why global illumination and why progressive methods

%\begin{figure}
%   \centering
%   \includegraphics{example.jpg} % requires the graphicx package
%   \caption{example caption}
%   \label{fig:example}
%\end{figure}
%
%   
%Co to je rendering
%Zobrazovaci rovnice distribuce energie v prostoru. 
%
%Co to je raytracing
%paprsek protina geometrii v zakladu 
%
%Jsou stale poupularnejsi progresivni metody renderingu, kdy se obraz pred stale vylepsuje, takze muzem videt nahled velice rychle.
%
%ray tracing se stava poupularnejsi a popularnejsi, mene narocny na nastavovani a fizikalne korektni. Navic sceny jsou slozitejsi a slozitejsi divaci narocnejsi. 
%
%Co to jsou volumetricke efekty a opticky aktivni prostredi
%
%Avsak volumetricke efekty se stale dost casto rasterizujou.
%
%Jak je to náročný a že se stále dost používají aproximační metody (rasterizace, deep shadow maps ) a nemodelují se náročné věci jako radiative transport surface to media volumetric caustics atd. Ani nástroje používané v produkci přímo nepodporuje a nebo jen v podobě náročných fotonových map, je to pomalé , takže artisti nemůžou odladit vse co chteji malo iteraci.



%\section{Podkapitola}
